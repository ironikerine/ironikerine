%------------------------------------
% Dario Taraborelli
% Typesetting your academic CV in LaTeX
%
% URL: http://nitens.org/taraborelli/cvtex
% DISCLAIMER: This template is provided for free and without any guarantee 
% that it will correctly compile on your system if you have a non-standard  
% configuration.
% Some rights reserved: http://creativecommons.org/licenses/by-sa/3.0/
%------------------------------------

%!TEX TS-program = xelatex
%!TEX encoding = UTF-8 Unicode

\documentclass[10pt, a4paper]{article}
\usepackage{fontspec} 
\usepackage{kotex}

%\setmainhangulfont{./fonts/NanumMyeongjo.ttf} % NanumMyeongjo
%\setsanshangulfont{./fonts/NanumMyeongjo.ttf}     % MalgunGothic
%\setmonohangulfont{./fonts/NanumGothic.ttf}

% DOCUMENT LAYOUT
\usepackage{geometry} 
\geometry{a4paper, textwidth=5.5in, textheight=8.5in, marginparsep=7pt, marginparwidth=.6in}
\setlength\parindent{0in}

% FONTS
\usepackage[usenames,dvipsnames]{xcolor}
\usepackage{xunicode}
\usepackage{xltxtra}
\usepackage{wrapfig}
\defaultfontfeatures{Mapping=tex-text}
%\setromanfont [Ligatures={Common}, Numbers={OldStyle}, Variant=01]{Linux Libertine O}
%\setmonofont[Scale=0.8]{Monaco}
%%% modified by Karol Kozioł for ShareLaTeX use
\setmainfont[
	  Ligatures={Common}, Numbers={OldStyle}, Variant=01,
	  BoldFont=LinLibertine_RB.otf,
	  ItalicFont=LinLibertine_RI.otf,
	  BoldItalicFont=LinLibertine_RBI.otf,Path=./fonts/
]{LinLibertine_R.otf}
\setmonofont[Scale=0.8,Path=./fonts/]{DejaVuSansMono.ttf}

% ---- CUSTOM COMMANDS
\chardef\&="E050
\newcommand{\html}[1]{\href{#1}{\scriptsize\textsc{[html]}}}
\newcommand{\pdf}[1]{\href{#1}{\scriptsize\textsc{[pdf]}}}
\newcommand{\doi}[1]{\href{#1}{\scriptsize\textsc{[doi]}}}
% ---- MARGIN YEARS
\usepackage{marginnote}
\newcommand{\amper{}}{\chardef\amper="E0BD }
\newcommand{\years}[1]{\marginnote{\scriptsize #1}}
\renewcommand*{\raggedleftmarginnote}{}
\setlength{\marginparsep}{7pt}
\reversemarginpar

% HEADINGS
\usepackage{sectsty} 
\usepackage[normalem]{ulem}
\sectionfont{\mdseries\upshape\Large}
\subsectionfont{\mdseries\scshape\normalsize} 
\subsubsectionfont{\mdseries\upshape\large} 

% PDF SETUP
% ---- FILL IN HERE THE DOC TITLE AND AUTHOR
\usepackage[%dvipdfm, 
bookmarks, colorlinks, breaklinks, 
% ---- FILL IN HERE THE TITLE AND AUTHOR
	pdftitle={CV},
	pdfauthor={Sung-Yong Kim},
	pdfproducer={Sung-Yong Kim}
]{hyperref}  
\hypersetup{linkcolor=blue,citecolor=blue,filecolor=black,urlcolor=MidnightBlue} 

% DOCUMENT
\begin{document}
%\begin{wrapfigure}{r}{3.5cm}
%	\vspace{-20pt}
%	\begin{center}
%	\includegraphics[width=3.3cm]{sykim}
%	\end{center}
%	\vspace{-200pt}
%\end{wrapfigure}
{\LARGE Sung-Yong Kim}\\[1cm]
School of Architecture\\
Changwon National University, Korea\\[.2cm]
Office: \texttt{82-55-213-3806}\\
Phone: \texttt{82-10-7656-1979}\\
Fax: \texttt{82-55-213-3809}\\[.2cm]
email: \href{mailto:sungyong.kim@changwon.ac.kr}{sungyong.kim@changwon.ac.kr}\\
ORCID: \href{https://orcid.org/0000-0002-0110-1546
}{https://orcid.org/0000-0002-0110-1546}\\ 
%\vfill
\hrule
\section*{Current Position}
\emph{Assistant Professor}, School of Architecture, Changwon National University

%\hrule
\section*{Research Area}
 Vibration Control • Seismic Design • Structural Health Monitoring\\
 Steel/Composite Structures • Fire Design% • Seismic Isolation

%\hrule
\section*{Education}
\noindent
\years{2002-2008}Bachelor, Seoul National University\\
\years{2009-2010}Master's Degree, Seoul National University\\
\years{2011-2017}Ph\,D, Seoul National University\\

%\hrule
\section*{Grants, Honors \& Awards}
\noindent
\years{2017}Key Scientific Article for the paper entitled \emph{Strain Compatibility for the Design of Short Rectangular Concrete Filled Tube Columns under Eccentric Axial Loads}. \\
\years{2014}2014 Sen Kuzo Scholarship

\section*{Projects}
\noindent
\subsection*{Research Director}
\noindent
\years{2020}에너지-내진성능~동시확보를~위한~열교차단파스너의~실물대실험, \texttt{교육부}\\
\years{2020}화재~발생한~건축물의~콘크리트~화해~측정기준에~대한~연구개발, \texttt{중소기업기술정보진흥원}\\
\years{2020-2023}지진력~추정~학습기술을~적용한~중소형~시설물~지진피해평가관리기술~개발, \texttt{한국산업기술평가관리원}\\
\years{2019-2021}장경간~건축물~지붕을~위한~트러스빔과~비대칭~강재빔을~적용한~합성보형~스트롱빔~시스템~개발, \texttt{중소기업기술정보진흥원}\\
\years{2018-2021}비대칭~Bouc-Wen~모형~개발~및~실시간~손상정도~산정~알고리즘~개발, \texttt{한국연구재단}\\
\years{2019}보령메디앙스 물류센터 진동측정, (주)\texttt{센구조연구소}\\
\years{2019}항공유 급유관로 안전진단(유한요소 모델링 해석), (주)\texttt{센구조연구소}\\

\subsection*{Participating Researcher}
\noindent
\years{2018-2018}5G~통신~Infra~내진모델~개발, (주) \texttt{SK Telecom}\\
\years{2018-2021}건축물~비구조요소~내진성능~확보기술~개발, \texttt{국토교통과학기술진흥원}\\
\years{2016-2018}TS880급~강재~대응~볼트접합부~및~강관접합부~적용성~평가, (주)\texttt{포스코}\\
\years{2013-2018}지진~및~기후변화~대응~소규모~기존~건축물~구조~안전성~향상기술~개발, \texttt{국토교통과학기술진흥원}\\
\years{2015-2016}동대구~복합환승센터현장~지상9층~안정성~및~진동검토~연구용역, (주)\texttt{신세계건설}\\
\years{2015-2015}진동방지~설계~및~감쇠기~설치, (주)\texttt{한라}\\
\years{2011-2014}고성능강의~건축현장~적용성~향상을~위한~기초~및~응용기술~연구, (재)\texttt{포항산업과학연구원}\\
\years{2014-2014}규모~8.3급의~강력한~지진에도~견디는~내진배전반~구현을~위한~배전반~받침대~소재~및~시스템~개발을~위한~연구, (주)\texttt{서전기전}\\
\years{2014-2014}규모~8.3급의~지진에~견디는~배전반~내진성능~확보를~위한~가새보강~시스템~개발, (주)\texttt{이레이티에스}\\
\years{2014-2014}New Ceiling System Development with Sheetrock Brand, \texttt{Boral Plasterboard System Co., Ltd}\\
\years{2013-2013}Sliding Step을~이용한~조립식~시스템~철골계단~개발, (주)\texttt{태영건설}\\
\years{2013-2013}콘크리트충전~각형강관기둥의~P-M조합강도~예측을~위한~구성방정식~개발, (주)\texttt{포스코}\\
\years{2009-2013}고강도강~설계/제작~지침~정립~및~합성부재~개발, \texttt{국토해양부}\\
\years{2010-2010}SRC기둥-무량판~접합부~시공~및~설계기법~개발, (재)\texttt{포항산업과학연구원}\\

\section*{Publications \& Talks}

\subsection*{Journal articles}
\noindent
\years{2021}Sung-Yong Kim, Cheol-Ho Lee (2021), Nondimensionalized Bouc–Wen model with structural degradation for Kalman filter–based real-time monitoring, \emph{Engineering Structures}, 244, 112674. \\
\years{2020}Sung-Yong Kim, Jaemin Kim (2020), Constrained Unscented Kalman Filter for Structural Identification of Bouc-Wen Hysteretic System, \emph{Advances in Civil Engineering}.\\
\years{2020}Sung-Yong Kim, Cheol-Ho Lee (2020), Analysis and Optimization of Multiple Tuned Mass Dampers with Coulomb-type Frictional Mechanism, \emph{Engineering Structures}, 209, 110011.\\
\years{2019}Cheol-Ho Lee, Jong-Hyun, Jung, Sung-Yong Kim (2019), Cyclic Seismic Performance of Weak-Axis RBS Welded Steel Moment Connections, \emph{International Journal of Steel Structures}, 19(2), pp.1-13.\\
\years{2019}Sung-Yong Kim, Cheol-Ho Lee (2019), Description of Asymmetric Hysteresis Behavior Based on the Bouc-Wen Model and Piecewise Linear Strength-Degradation Functions, \emph{Engineering Structures}, 181, pp.181-191.\\
\years{2019}Sung-Yong Kim, Cheol-Ho Lee (2019), Peak Response of Frictional Tuned Mass Dampers Optimally Designed to White-Noise Base Acceleration, \emph{Mechanical Systems and Signal Processing}, 117, pp.319-332.\\
\years{2018}Sung-Yong Kim, Cheol-Ho Lee (2018), Optimal Design of Linear Multiple Tuned Mass Dampers Subjected to White-Noise Base Acceleration Considering Practical Configurations, \emph{Engineering Structures}, 171, pp.516-528.\\
\years{2017}Sung-Yong Kim, Cheol-Ho Lee (2017), Seismic Retrofit of Welded Steel Moment Connections with Highly Composite Floor Slabs, \emph{Journal of Constructional Steel Research}, 139, pp.62-68.\\
\years{2016}Sung-Yong Kim, Cheol-Ho Lee, Na-Eun Kim (2016), Effective Impulse Model for Prediction of Vibration Response of High-Frequency Steel Staircases, \emph{Journal of Constructional Steel Research}, 126, pp.129-138.\\
\years{2016}Cheol-Ho Lee, Jong-Hyun Jung, Sung-Yong Kim, Jeong-Jae Kim (2016), Investigation of Composite Slab Effect on Seismic Performance of Steel Moment Connections, \emph{Journal of Constructional Steel Research}, 117, pp.91-100.\\
\years{2016}Cheol-Ho Lee, Thomas H.-K Kang, Sung-Yong Kim, Kiyong Kang (2016), Strain Compatibility Method of the Design of Short Rectangular Concrete-Filled Tube Columns under Eccentric Axial Loads, \emph{Construction and Building Materials}, 121, pp.143-153.\\
\vspace{5mm}\\
\years{2018}이철호, 김성용, 박지훈, 김동관, 김태진, 박경훈 (2018), 9.12 경주지진~및~11.15~포항지진의~구조손상포텐셜~비교연구, \texttt{한국지진공학회~논문집}, 한국지진공학회, 22(3): 175-184.\\
\years{2017}이철호, 김성용 (2017), 바닥슬래브를~고려한~용접철골모멘트접합부의~내진보강, \texttt{한국강구조학회~논문집}, 한국강구조학회, 29(1): 25-36.\\
\years{2016}이철호, 박지훈, 김태진, 김성용, 김동관 (2016), 2016년~9월~12일~M5.8~경주지진의~데미지~포텐셜~분석~및~내진공학~측면의~시사점, \texttt{한국지진공학회~논문집}, 한국지진공학회, 20(7): 527-536.\\
\years{2015}이철호, 정종현, 김성용 (2015), RBS 약축~용접모멘트접합부의~내진성능~평가, \texttt{한국강구조학회~논문집}, 한국강구조학회 17: 549–560.\\
\years{2015}김나은, 이철호, 김성용 (2015), 고진동수~계단의~진동응답~산정을~위한~등가임펄스~산정식~제안, \texttt{한국강구조학회~논문집}, 한국강구조학회, 17: 891–921.\\
\years{2015}이철호, 강기용, 김성용 (2015), 콘크리트충전~각형강관단주의~P-M 조합강도~예측을~위한~콘크리트 구성방정식, \texttt{한국강구조학회~논문집}, 한국강구조학회, 18: 639–641.\\
\years{2014}김성용, 이철호, 김나은, 조성상, 정운옥 (2014), 실물대~목업실험에~의한~슬라이딩스텝~철골계단의~진동~및~구조성능 평가, \texttt{한국강구조학회~논문집}, 한국강구조학회, : 844–847\\
\years{2014}고아라, 이철호, 김성용 (2014), 비대칭~동조질량감쇠기를~활용한~바닥진동제어의~강건성~향상~방안, \texttt{한국강구조학회~논문집}, 한국강구조학회,\\
\years{2013}이철호, 강기용, 김성용, 구철회 (2013), 각형~콘크리트충전~강관기둥~부재의~구조설계기준~비교연구, \texttt{한국강구조학회~논문집}, 한국강구조학회, 18: 121–128\\
\years{2012}김성용, 이철호 (2012), 개인별~시간지연효과를~고려한~확률론적~군중~하중모형~개발, \texttt{한국강구조학회~논문집}, 한국강구조학회, 

\subsection*{Conferences (International Only)}
\noindent
\years{2017}Cheol-Ho Lee, Sung-Yong Kim (2017), Seismic Retrofit of Welded Steel Moment Connections with Highly Composite Slabs, \emph{16th World Conference on Earthquake Engineering}, Santiago, Chile.\\
\years{2015}Sung-Yong Kim, Cheol-Ho Lee (2015) Modal Properties and Seismic Application of Asymmetric Tuned Mass Dampers, \emph{Joint Seminar on Earthquake Engineering for Building Structures (SEEBUS)}, Kyoto, Japan.\\
\years{2014}Sung-Yong Kim, Cheol-Ho Lee (2014) Enhancing Robustness of Floor Vibration Control by using Asymmetric Tuned Mass Damper, \emph{6th World Conference on Structural Control and Monitoring}, Barcelona, Spain.\\
\years{2014}Sung-Yong Kim, Cheol-Ho Lee (2014) Mitigation of Off-tuning Effect of Floor Vibration by Using Asymmetric Tuned Mass Damper, \emph{SEEBUS}, Seoul, Korea. \\
\years{2013}Sung-Yong Kim, Cheol-Ho Lee (2013) Development of Time Lag Considered Crowd Load Model Based on Spectral Density Approach, \emph{Proceedings of the 7th International Symposium on Steel Structures}, Jeju, Korea.

\subsection*{Newspaper Articles}
\noindent
\years{2011}이철호, 김성용 (2011-12), 유럽지역~바닥진동의~새로운~설계법, \texttt{한국강구조학회지}.

\section*{Teaching}
\years{2019-Now}Architectural Material Mechanics, Changwon National University.\\
\years{2019-Now}RC and Steel, Changwon National University.\\
\years{2019-Now}Architectural Structural Analysis, Changwon National University.\\
\years{2019-Now}Architectural Structural Mechanics, Changwon National University.\\
\years{2018}General Mathematics 1, Dankook University.\\
\years{2018}Statics 2, Dankook University.\\
\years{2017}Engineering Mathematics 2, Dankook University.\\
\years{2017}Steel Structures 1, Dankook University.\\
\years{2017}Building Systems, Sejong University.\\
\years{2013}Engineering Mathematics 1, Dankook University.
%\hrule

%\vspace{1cm}
\vfill{}
%\hrulefill

\begin{center}
{\scriptsize  Last updated: \today}%\- •\- 
% ---- PLEASE LEAVE THIS BACKLINK FOR ATTRIBUTION AS PER CC-LICENSE
%Typeset in \href{http://nitens.org/taraborelli/cvtex}{
%\fontspec{Times New Roman}
%\XeTeX }\\
% ---- FILL IN THE FULL URL TO YOUR CV HERE
%\href{http://nitens.org/taraborelli/cvtex}{http://nitens.org/taraborelli/cvtex}}
\end{center}

\end{document}